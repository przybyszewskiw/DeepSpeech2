%
% Niniejszy plik stanowi przykład formatowania pracy magisterskiej na
% Wydziale MIM UW.  Szkielet użytych poleceń można wykorzystywać do
% woli, np. formatujac wlasna prace.
%
% Zawartosc merytoryczna stanowi oryginalnosiagniecie
% naukowosciowe Marcina Wolinskiego.  Wszelkie prawa zastrzeżone.
%
% Copyright (c) 2001 by Marcin Woliński <M.Wolinski@gust.org.pl>
% Poprawki spowodowane zmianami przepisów - Marcin Szczuka, 1.10.2004
% Poprawki spowodowane zmianami przepisow i ujednolicenie 
% - Seweryn Karłowicz, 05.05.2006
% Dodanie wielu autorów i tłumaczenia na angielski - Kuba Pochrybniak, 29.11.2016

% dodaj opcję [licencjacka] dla pracy licencjackiej
% dodaj opcję [en] dla wersji angielskiej (mogą być obie: [licencjacka,en])
\documentclass[licencjacka,en]{pracamgr}
\usepackage{hyperref}
\newcommand{\bibDownloadDate}{\today}


% Dane magistrantów:
\autor{Piotr Ambroszczyk}{385090}
\autori{Łukasz Kondraciuk}{385775}
\autorii{Wojciech Przybyszewski}{386044}
\autoriii{Jan Tabaszewski}{386319}

\title{NVIDIA Deep Speech}
\titlepl{NVIDIA Deep Speech}

%\tytulang{An implementation of a difference blabalizer based on the theory of $\sigma$ -- $\rho$ phetors}

%kierunek: 
% - matematyka, informacyka, ...
% - Mathematics, Computer Science, ...
\kierunek{Computer Science}

% informatyka - nie okreslamy zakresu (opcja zakomentowana)
% matematyka - zakres moze pozostac nieokreslony,
% a jesli ma byc okreslony dla pracy mgr,
% to przyjmuje jedna z wartosci:
% {metod matematycznych w finansach}
% {metod matematycznych w ubezpieczeniach}
% {matematyki stosowanej}
% {nauczania matematyki}
% Dla pracy licencjackiej mamy natomiast
% mozliwosc wpisania takiej wartosci zakresu:
% {Jednoczesnych Studiow Ekonomiczno--Matematycznych}

% \zakres{Tu wpisac, jesli trzeba, jedna z opcji podanych wyzej}

% Praca wykonana pod kierunkiem:
% (podać tytuł/stopień imię i nazwisko opiekuna
% Instytut
% ew. Wydział ew. Uczelnia (jeżeli nie MIM UW))
\opiekun{dr Janina Mincer-Daszkiewicz \\
  Instytut Informatyki\\
}

% miesiąc i~rok:
\date{May 2019}

%Podać dziedzinę wg klasyfikacji Socrates-Erasmus:
\dziedzina{ 
%11.0 Matematyka, Informatyka:\\ 
%11.1 Matematyka\\ 
%11.2 Statystyka\\ 
11.3 Informatyka\\ 
%11.4 Sztuczna inteligencja\\ 
%11.5 Nauki aktuarialne\\
%11.9 Inne nauki matematyczne i informatyczne
}

%Klasyfikacja tematyczna wedlug AMS (matematyka) lub ACM (informatyka)
\klasyfikacja{D. Software\\
  %D.127. Blabalgorithms\\
  %D.127.6. Numerical blabalysis
  }

% Słowa kluczowe:
\keywords{Deep Speech, ASR, Neural Networks, Machine Learning, Python, PyTorch, NVIDIA, RNN, multi-GPU, FP16}

% Tu jest dobre miejsce na Twoje własne makra i~środowiska:
\newtheorem{defi}{Definicja}[section]

% koniec definicji

\begin{document}
\maketitle

%tu idzie streszczenie na strone poczatkowa
\begin{abstract}
  The authors of this thesis focus on implementing scripts for training DeepSpeech2 model for Automatic Speech Recognition. We try to reproduce results obtained by Baidu Research in End-to-End Speech Recognition paper DODAC DOKUMENTACJE using \texttt{PyTorch} framework. We also experiment with obtaining dataset for Polish language and trying DeepSpeech2 model for it. Finally, we provide fully trained models for English and Polish together with statistics about how changing hyperparametrs and architecture impacts model's performance and accuracy.
\end{abstract}

\tableofcontents
%\listoffigures
%\listoftables

\chapter*{Introduction}
\addcontentsline{toc}{chapter}{Introduction}
The goal of our thesis is to implement Automatic Speech Recognition (ASR) model DeepSpeech2 described in \cite{DS2}. We will use  \texttt{PyTorch} deep learning framework, which is supported with CUDA, and is considered to be comfortable to work with. Authors of DeepSpeech2 prepared it only for recognizing English and Mandarin, so one of goals of our thesis is to experiment with applying it to Polish language as well. We think, it is going to be the biggest challenge, since accuracy of the model depends not only on its implementation, but also on the size and diversity of used dataset. Therefore we plan to find appropriate one (paying attention to licenses and copyrights) and prepare it adequately. In \cite{DS1} the authors present techniques for dataset augmentation by e.g. injecting noise. That's because working with specially prepared, clear vocal sounds does not indicate success with processing words and statements of lower quality, and best model's learning parameters can vary in different languages. We are going to use these techniques to improve our model robustness. The great size of dataset creates another problem -- we need our model to be able to train on that data in reasonable time and then work in the real time. To achieve this we will have to train our model on GPU, bearing in mind that \texttt{PyTorch} framework supports usage of CUDA accelerators. Moreover we plan to use open-source libraries prepared by NVIDIA which make it possible to train one neural network over multiple GPUs. Another optimization which speeds computations up is using half precision floating point numbers (also known as FP16) instead of single precision. \\


\chapter{Basic model description}\label{r:desc}

\chapter{Additional extensions}\label{r:extens}


\chapter{Experiments on architecture and hyperparameters}\label{r:hypers}


\chapter{Recognizing Polish language}\label{r:polish}

\section{Preparing Polish dataset}
\section{Model’s architecture description}
\section{Comparison of model's performance on English and Polish}


\chapter{Conclusions}

To sum up, we present \texttt{PyTorch} scripts for training DeepSpeech2 model for ASR. We also present already trained models for English and Polish as well as the results of our experiments justifying using specific hyperparameters and architecture solutions.\\

\begin{thebibliography}{99}
\addcontentsline{toc}{chapter}{Bibliography} 

\bibitem{DS1} Hannun et al. 
\textit{Deep Speech: Scaling up end-to-end speech recognition}, Silicon Valley AI Lab 2014, \href{https://arxiv.org/abs/1412.5567}{https://arxiv.org/abs/1412.5567}
  
\bibitem{DS2} Baidu Research \textit{ Deep Speech 2: End-to-End Speech Recognition in English and Mandarin}, Silicon Valley AI Lab 2015, \href{https://arxiv.org/abs/1512.02595}{https://arxiv.org/abs/1512.02595}
  
\end{thebibliography}
All the files were downloaded on \bibDownloadDate

\end{document}


%%% Local Variables:
%%% mode: latex
%%% TeX-master: t
%%% coding: latin-2
%%% End:
